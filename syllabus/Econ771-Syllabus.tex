\documentclass{article}
\usepackage{graphicx,amssymb,amsmath,pstricks,setspace}
\usepackage[left=2cm,right=2cm,top=1.5cm,bottom=2cm]{geometry}
\usepackage{fancyhdr}
\usepackage{verbatim}
\usepackage{longtable}
\usepackage{array}
\usepackage[round]{natbib}
\renewcommand{\refname}{Reading List}
\setstretch{1.2}

\begin{document}

\thispagestyle{empty}

\vspace{-.5in}

\begin{center}
\textsc{\Large{Econ 771: Health Economics II}} \\
\textsc{\large{Fall 2020}}
\end{center}

\noindent \textsc{Professor:} Ian McCarthy\\
\noindent \textsc{Class:} TTH 8:00 -- 10:00am \\
\noindent \textsc{Room:} Rich 103 \\
\noindent \textsc{Office:} Rich 319 \\
\noindent \textsc{Office Hours:} TTH 10:00 -- 11:30am \\
\noindent \textsc{Email:} ian.mccarthy@emory.edu \\

\vspace{-.25in}

\section*{Course Description}
This course explores the industrial organization of health care markets in the U.S. We will focus on the following areas: hospital ownership and production, physician agency, quality disclosure, and hospital competition. If we have time, we will briefly cover some issues on health insurance markets. The class is effectively designed a combination of empirical IO and causal inference, with applications to healthcare. As such, we will also examine several econometric tools and causal inference identification strategies. These methods will be introduced as needed throughout the course.

Supply-side health economics is a rapidly growing field with many new developments, particularly in the areas of bargaining in two-sided markets and physician learning. Some of these recent developments utilize tools from network analysis and machine learning, which we unfortunately do not have time to cover in this course. I've also chosen specific topics that overlap most with my own research -- the upside here is that I can speak somewhat confidently about the literature and empirical studies in this area, but the downside is that some very interesting areas of health economics are not studied. For example, we will largely ignore issues of the prescription drug market, medical devices, and physician labor supply. My hope is that the content that we do cover will provide a springboard for those interested in these other important areas.

\section*{Course Objectives}
I have four central goals of this course:
\begin{enumerate}
 \item Understand pressing issues in the field of supply-side health economics
 \item Be able to read and apply standard causal inference techniques in the area of healthcare
 \item Provide constructive criticism of academic work in this area
 \item Develop your own preliminary research in some area of health economics
\end{enumerate}

\section*{Text and Other Materials}
As an Economics PhD course, we will rely on academic papers from the reading list below. I expect everyone to read the papers in advance and come to class with questions on the study's contribution, empirical techniques, identification strategies, and datasets used. Where possible, we will work to replicate results or apply the paper's central methodology to some simulated data. My goal with each paper is to discuss the analysis in detail. As such, the primary reading list is perhaps shorter than a standard PhD course. I've provided supplemental reading in each section for those interested in additional readings in a specific area.

\section*{Course Outline}
Below is a preliminary outline of specific topics and readings throughout the semester. Based on our collective interests, discussions, and timing, individual papers and order may change somewhat, but any such changes will be announced in advance. For each class, one of us will present the ``primary reading,'' followed by a general discussion of the paper and topic among the class. Where relevant, we will work through some simulation exercises or derive some theoretical results. In general, papers listed as ``supplemental'' are other empirical applications related to the topic of that day, while papers listed as ``background'' reading are either survey articles, purely theoretical articles, or earlier seminal work on that topic. Days with ** indicate that no one will present a paper on that day. Those days are reserved for some basic theoretical derivations or presentation of new empirical methods.

\begin{longtable}{lp{11cm}}
  Class 1 & Introduction to the economics of health care  \\
          \multicolumn{1}{r}{General references:} & \cite{arrow1963}, \cite{dranove2000}, \cite{evans2000}, \cite{morrisey2008}, \cite{gaynor2015jel} \\
  \hline
  \multicolumn{2}{c}{\textbf{The hospital objective function(s) and financial incentives}} \\
  \hline\hline
  Class 2 & Theoretical models of not-for-profits**  \\
        \multicolumn{1}{r}{Primary Reading:} & \cite{newhouse1970}, \cite{pauly1978}, \cite{dranove1988} \\
        \multicolumn{l}{r}{Background:} & \cite{sloan2000} \\
  \hline
  Class 3 & Empirical evidence on ownership type \\
        \multicolumn{1}{r}{Primary Reading:} & \cite{duggan2000}, \cite{horwitz2009}, \cite{chang2017} \\
        \multicolumn{1}{r}{Supplemental:} & \cite{frank1991}, \cite{gruber1994}, \cite{sloan2001}, \cite{deneffe2002}, \cite{duggan2002}, \cite{kessler2002}, \cite{gaynor2003}, \cite{david2009}, \cite{bayindir2012} \\
  \hline
  Class 4 & Response to Medicare payment changes \\
          \multicolumn{1}{r}{Primary Reading:} & \cite{dafny2005}, \cite{dranove2019}  \\
          \multicolumn{1}{r}{Supplemental:} & \cite{cook2020}  \\
  \hline
  Class 5 & Hospital cost-shifting \\
          \multicolumn{1}{r}{Primary Reading:} & \cite{dranove2017} \\
          \multicolumn{1}{r}{Supplemental:} & \cite{zwanziger2000}, \cite{wu2010}, \cite{darden2018}, \cite{david2014}  \\
          \multicolumn{1}{r}{Background:} & \cite{hay1983}, \cite{dranove1988}, \cite{frakt2011} \\
  \hline
  Class 6 & Response to insurance policies \\
          \multicolumn{1}{r}{Primary Reading:} & \cite{batty2017}, \cite{eliason2018}, \cite{geruso2020}  \\
          \multicolumn{1}{r}Supplemental:} & \cite{lee2020} \\
  \hline
  \multicolumn{2}{c}{\textbf{Physician agency and treatment decisions}} \\
  \hline\hline
  Class 7 & The agency problem in healthcare** \\
        \multicolumn{1}{r}{Primary Reading:} & \cite{mcguire2000} \\
        \multicolumn{1}{r}{Supplemental:} & \cite{chandra2011} \\
        \multicolumn{1}{r}{Background:} &  \cite{grossman1983} \\
  \hline
  Class 8 & Empirical evidence of physician agency \\
        \multicolumn{1}{r}{Primary Reading:} & \cite{gruber1996}, \cite{clemens2017} \\
        \multicolumn{1}{r}{Supplemental:} & \cite{clemens2014}, \cite{baker2016}, \cite{gaynor2016}, \cite{beckert2018} \\
  \hline 
  Class 9 & The norms hypothesis and treatment decisions across patients \\
        \multicolumn{1}{r}{Primary Reading:} & \cite{chandra2007}, \cite{baicker2013} \\
        \multicolumn{1}{r}{Supplemental:} & \cite{mcguire1991}, \cite{glied2002}, \cite{baker2003}  \\
        \multicolumn{1}{r}{Background:} & \cite{newhouse1978}, \cite{frank2007}, \cite{chernew2010}, \cite{landon2017} \\

  



  \hline
  \multicolumn{2}{c}{\textbf{Hospital Competition and Healthcare Markets}} \\
  \hline\hline
  Class 7, 9/20 & Competition in price and quality \\
          \multicolumn{1}{r}{Primary Reading:} & \cite{fan2013}, \cite{lewis2016} \\
          \multicolumn{1}{r}{Background:} & \cite{spence1975}, \cite{mussa1978}, \cite{hausman1994}, \cite{dana2011} \\
  \hline
  Class 8, 9/25 & Measuring hospital market power \\
          \multicolumn{1}{r}{Primary Reading:} & \cite{dranove2016} \\
          \multicolumn{1}{r}{Supplemental:} & \cite{capps2003} \\
          \multicolumn{1}{r}{Background:} & \cite{dranove1990} \\
  \hline
  Class 9, 9/27 & Evidence from ``structure-conduct-performance'' studies \\
          \multicolumn{1}{r}{Primary Reading:} & \cite{kessler2000}, \cite{cooper2017} \\
          \multicolumn{1}{r}{Supplemental:} & \cite{dranove1992hospital}, \cite{lynk1995}, \cite{keeler1999} \\
  \hline
  Class 10-11, 10/2, 10/4 & Estimating (static) discrete choice models** \\
          \multicolumn{1}{r}{Primary Reading:} & \cite{nevo2000}, \cite{berry1994} \\
          \multicolumn{1}{r}{Supplemental:} & \cite{berry1995}, \cite{berry2004} \\
          \multicolumn{1}{r}{Background:} & \cite{deaton1980}, \cite{train2009} \\
  \hline
  Class 12, 10/11 & Evidence from mergers/closures \\
          \multicolumn{1}{r}{Primary Reading:} & \cite{dafny2009}, \cite{craig2018} \\
          \multicolumn{1}{r}{Supplemental:} & \cite{vita2001}, \cite{gaynor2003}, \cite{lindrooth2003}  \\
          \multicolumn{1}{r}{Background:} & \cite{gaynor2012rwjf} \\
  \hline
  \multicolumn{2}{c}{\textbf{Information Asymmetries}} \\
  \hline\hline
  Class 15, 10/23 & Role of quality and price disclosure \\
          \multicolumn{1}{r}{Primary Reading:} &  \cite{dranove1992}, \cite{abaluck2011} \\
          \multicolumn{1}{r}{Supplemental:} & \cite{farrell2007}, \cite{ketcham2012}, \cite{handel2013}, \cite{ericson2014} \\
          \multicolumn{1}{r}{Background:} &  \cite{dranove2010} \\
  \hline
  Class 16, 10/25 & Effects of quality and price disclosure in healthcare \\
          \multicolumn{1}{r}{Primary Reading:} & \cite{dranove2003more}, \cite{dafny2008}, \cite{cooper2018} \ \\
          \multicolumn{1}{r}{Supplemental:} &  \cite{chernew1998}, \cite{scanlon2002}, \cite{wedig2002}, \cite{jin2006}, \cite{reid2013}, \cite{darden2015}, \cite{mccarthy2017} \\

  \hline
  \multicolumn{2}{c}{\textbf{Physician-hospital Vertical Integration}} \\
  \hline\hline
  Class 17, 10/30 & What is Integration?** \\
          \multicolumn{1}{r}{Primary Reading:} & \cite{burns2002} \\
          \multicolumn{1}{r}{Background:} & \cite{kocher2011}, \cite{post2017}  \\
  \hline
  Class 18, 11/1 & Effects on Hospitals \\
          \multicolumn{1}{r}{Primary Reading:} & \cite{baker2014}, \cite{lin2017} \\
          \multicolumn{1}{r}{Supplemental:} & \cite{ciliberto2006}, \cite{cuellar2006}, \cite{neprash2015}, \cite{konetzka2018} \\
  \hline
  Class 19, 11/6 & Effects on Physicians \\
          \multicolumn{1}{r}{Primary Reading:} & \cite{baker2016}, \cite{koch2017} \\
  \hline
  \multicolumn{2}{c}{\textbf{Health Insurance Markets}} \\
  \hline\hline
  Class 20, 11/13 & Insurance Markets and Competitiveness \\
          \multicolumn{1}{r}{Primary Reading:} &  \cite{dafny2010}, \cite{dafny2012}, \cite{bundorf2012}  \\
          \multicolumn{1}{r}{Supplemental:} & \cite{town2003}, \cite{einav2010}, \cite{einav2010qje}, \cite{starc2014} \\
  \hline
  Class 21, 11/15 & Adverse Selection \\
          \multicolumn{1}{r}{Primary Reading:} & \cite{cutler1998},  \cite{decarolis2017} \\
          \multicolumn{1}{r}{Background:} & \cite{ackerloff1970}, \cite{frank2000}, \cite{rothschild1976}, \cite{einav2011} \\
  \hline
  Class 22, 11/16 & Moral Hazard \\
          \multicolumn{1}{r}{Primary Reading:} &  \cite{einav2013} \\
          \multicolumn{1}{r}{Supplemental:} & \cite{manning1996} \\
          \multicolumn{1}{r}{Background:} & \cite{finkelstein2014}, \cite{einav2017} \\
  \hline
  Class 23, 11/27 & Managed Competition \\
          \multicolumn{1}{r}{Primary Reading:} &  \cite{curto2015}, \cite{einav2015} \\
          \multicolumn{1}{r}{Supplemental:} & \cite{song2013}, \cite{cabral2014}, \cite{stockley2014}, \cite{duggan2016}, \cite{pelech2018} \\
  \hline
  \multicolumn{2}{c}{\textbf{Hospital and Insurer Bargaining}} \\
  \hline\hline
  Class 24-25, 11/29, 12/4 & Estimating (static) structural IO models** \\
          \multicolumn{1}{r}{Primary Reading:} & \cite{reiss2007} \\
  \hline
  Class 26, 12/6 & Structural approaches in healthcare \\
          \multicolumn{1}{r}{Primary Reading:} & \cite{abraham2007}   \\
          \multicolumn{1}{r}{Supplemental:} & \cite{town2003} \\
          \multicolumn{1}{r}{Background:} & \cite{bresnahan1991} \\
  \hline
  Class 27, 12/11 & Applications of Nash Bargaining model \\
          \multicolumn{1}{r}{Primary Reading: } &  \cite{ho2017} \\
          \multicolumn{1}{r}{Supplemental:} & \cite{gowrisankaran2015}, \cite{lewis2015}
\end{longtable}


\pagebreak
\section*{Assignments}

\noindent \textbf{Presentations} \\
You will present two papers throughout the course of the semester. You can pick any paper listed as ``primary reading'' for any class day (other than days indicated with a **). Each presentation should be no more than 45 minutes, followed by questions and class discussion. The presentation should follow a standard conference setup, with a brief introduction/motivation, a very brief discussion of the literature and some context of the paper, discussion of the data, empirical analysis, and results. There is rarely enough time to discuss all of the sensitivity analyses or robustness checks in a 45 minute presentation.

Note that a presentation is not just a re-hashing of the paper in slide form. A good academic presentation should have as little information as possible on each slide, and the content on the slides doesn't necessarily need to follow that of the paper. For example, in a real-time environment, it is much easier to move between different aspects of the empirical analysis and data.

In addition to the presentation, please send me your slides in advance. For people in the economics department, I expect the slides to be completed in Beamer (LaTeX). Others can use PowerPoint, Prezi, or some other presentation tool as they see fit. My only recommendation with those other programs is that, for some of them like Prezi or Powtoon, it is easy for the presentation to become distracting. Slides should complement your presentation and not replace your presence!

Grades for each presentation will be based 50\% on the slides and 50\% on the actual presentation (delivery, clarity/organization, and content).

\medskip
\noindent \textbf{Referee Report} \\
Everyone must write a referee report based on an unpublished manuscript, along with a ``letter to the editor.'' I will provide a set of papers from which you can choose to write a report. You are also free to select your own, but you must get approval from me on the paper first.

For some advice on a good referee report, please read \cite{berk2017}. The letter to the editor should be submitted on Emory letterhead and signed (i.e., an official letter). It should contain a very brief summary of the report, and reference the report for more detail, as well as your final suggestion regarding revision, rejection, etc. Note that the letter to the editor is typically confidential, so you can offer more blunt criticism if necessary, but you should be careful with the tone of the letter and the report. The goal is to be constructive and try to help the editor in this decision. Remember, your conclusions can influence people's careers...there's not need to be rude.

Your referee report and letter to the editor are due on Tuesday, October 23rd.

\medskip
\noindent \textbf{Replication} \\
You must replicate the empirical analysis of a recent health economics paper. Your final product should consist of a discussion of the data sources, steps taken to clean the data and replicate the final analytic sample, a summary of your model and estimator, and your results. You should also include your code files (Stata, R, Python, SAS, etc.). Please ``turn in'' your replication analysis electronically, either via DropBox, Google Drive, some other cloud platform, or simply as a zipped attachment to an email.

Please organize your folders in a useful way. The way I organize things (thought certainly not the only way to do it) is to keep a folder for each new project and named accordingly. I typically have the following subfolders:
\begin{itemize}
    \item Data: This is where I keep the raw data files and any additional data files I create as part of my analysis. I also keep a ``Research Data'' folder on my computer that has raw data that I access regularly, in which case the ``Data'' folder in any given project is just a cleaned version of the larger files.
    \item Analysis: This is where I keep my Stata code files, log files, and results. I typically have a subfolder for the log files since I create one every time I run a code file as well as a subfolder for results.
    \item Manuscript: This is where I keep all of my files for the actual paper. I usually have a different subfolder for every month or two (whenever I create a sufficiently different version of the paper).
    \item Presentation: This is where I keep my slides and Beamer code files.
\end{itemize}
It's good to start developing some organization practices that work best for you. It's extremely easy to forget what you were doing on a project once you have several things going at once, especially when you wait for 6-8 months after submitting a paper for publication. The last thing you want is to not be able to replicate your own work!

You need to get approval from me on a paper to replicate by Friday, September 28. Whatever paper you select, the data need to be publicly available. For this reason, you may find it easer to replicate one of my Medicare Advantage papers. Final replications are due on Tuesday, November 13.

\medskip
\noindent \textbf{Literature Review} (for 2nd year students) \\
The literature review can be on any health economics topic of your choice, subject to my approval.  There is no specific page requirement.  10 double-spaced pages might be a good target, but an efficiently-written paper could be shorter, while a student wishing to use the paper as a springboard to a dissertation may choose to write more.  The paper will mostly consist of discussions of prior research, but should end with discussions of three open questions in the literature plus a proposed strategy for answering at least one of these questions.  The paper is due on the last day of class, December 11.

\medskip
\noindent \textbf{Draft Paper} (for 3rd year students) \\
The paper requirement is more extensive for 3rd year students. In this case, your papers should include not only an extensive literature review but also a preliminary empirical analysis, including a discussion of the data you're using, the construction of your sample, your identification strategy, econometric model and estimator, and some preliminary results. There is again no specific page requirement, but I expect at least 20 double-spaced pages to appropriately discuss your topic, context, data, and early results. Your draft paper should end with an outline of additional analyses you hope to run (i.e., robustness/sensitivity analyses, a discussion of different mechanisms of interest, policy-relevant heterogeneities in your estimated effects, etc.). The paper is due on the last day of class, December 11.


\section*{Evaluation}
Final grades will be determined as follows:
\begin{table}[h!]
\centerline{
\begin{tabular}{l|l}
 Opportunity & Percent \\
 \hline
 Presentations   & 20\% (10\% each) \\
 Referee Report  & 20\%  \\
 Replication     & 30\% \\
 Literature Review/Paper  & 30\%
\end{tabular}}
\end{table}

\section*{Course Policies}
Similar to a movie theater, we have a strict no cell phone, no computer policy in class. Please put your phones in airplane mode when you get to class, and I will do the same. The purpose of this policy is twofold: 1) these types of devices are extremely useful but also extremely distracting, and in my experience, I've found that our discussions are much more engaging and informative when we avoid these distractions; and 2) this is really just an issue of mutual respect, both for our time in class as well as your classmates. The more we can be engaged and respectful of one another, the more we'll enjoy the class.

\section*{Academic Integrity and Honor Code}
The Emory University Honor Code is taken seriously and governs all work in this course. Details about the Honor Code are available in the Laney Graduate School Handbook. By taking this course, you affirm that it is a violation of the code to plagiarize, to deviate from the instructions about collaboration on work that is submitted for grades, to give false information to a faculty member, and to undertake any other form of academic misconduct. You also affirm that if you witness others violating the code you have a duty to report them to the honor council.

\pagebreak
\bibliographystyle{authordate1}
\bibliography{BibTeX_Library}

\end{document}
